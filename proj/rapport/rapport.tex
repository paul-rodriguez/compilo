\documentclass[a4paper,10pt]{article}
\usepackage[utf8]{inputenc}
\usepackage[francais]{babel}
\usepackage{tikz}
\usetikzlibrary{automata,positioning}
\usepackage{amssymb}
\usepackage{hyperref}
\usepackage{vmargin}
\usepackage{pbox}
\usepackage{ltxtable, tabularx, longtable}
\usepackage{lscape}

\setmarginsrb%
 {2cm} %{leftmargin}
 {2.5cm} %{topmargin}
 {2.5cm} %{rightmarg}
 {2.5cm} %{bottommargin}
 {0cm}   %{headheight}
 {0cm}   %{headsep}
 {1.5cm} %{footheight}
 {1.5cm} %{footsep}

\hypersetup{
	unicode=true,         	% non-Latin characters in Acrobat�s bookmarks
	pdftoolbar=true,        % show Acrobat�s toolbar?
	pdfmenubar=true,        % show Acrobat�s menu?
	pdffitwindow=false,     % window fit to page when opened
	pdfstartview={FitH},    % fits the width of the page to the window
	pdftitle={Projet : Compilateur Perl},  
    pdfauthor={Rodriguez Paul, Vaccari Eric},	
    pdfnewwindow=true,      % links in new window
    colorlinks=true,       % false: boxed links; true: colored links
    linkcolor=black,          % color of internal links
    citecolor=green,        % color of links to bibliography
    filecolor=magenta,      % color of file links
    urlcolor=cyan           % color of external links
}

\title{Project INFO F 403 : Compilateur Perl}
\author{RODRIGUEZ Paul, VACCARI Eric}

\begin{document}

\maketitle
\pagebreak
\tableofcontents
\pagebreak

\section{Unités lexicales}
	\subsection{Tableau}
		\begin{center}
		    \begin{tabular}{ | l | l |}
		    	\hline Nom & Regex \\
		    	\hline var & \verb@ $[a-zA-Z_][a-zA-Z0-9_]* @ \\
		    	\hline identifier & \verb@ [a-zA-Z_][a-zA-Z0-9_]* @ \\
		    	\hline integer & \verb@ [0-9]+ @ \\
		    	\hline float & \verb@ {integer}\.{integer} @ \\
		    	\hline string & \verb@ '[^']*' @ \\
		    	\hline space & \verb@ [\t\n ] @ \\
		    	\hline comment & \verb@ #.*\n @ \\
		    	\hline lbrace & \verb@ \{ @ \\
		    	\hline rbrace & \verb@ \} @ \\
		    	\hline lpar & \verb@ \( @ \\
		    	\hline rpar & \verb@ \) @ \\
		    	\hline semicolon & \verb@ ; @ \\
		    	\hline call\_mark & \verb@ & @ \\
		    	\hline plus & \verb@ \+ @ \\
		    	\hline minus & \verb@ \- @ \\
		    	\hline times & \verb@ \* @ \\
		    	\hline divide & \verb@ \/ @ \\
		    	\hline not & \verb@ ! @ \\
		    	\hline notletters & \verb@ not @ \\
		    	\hline lazy\_and & \verb@ && @ \\
		    	\hline lazy\_or & \verb@ || @ \\
		    	\hline equals & \verb@ == @ \\
		    	\hline eq & \verb@ eq @ \\
		    	\hline different & \verb@ != @ \\
		    	\hline ne & \verb@ ne @ \\
		    	\hline lower & \verb@ < @ \\
		    	\hline lt & \verb@ lt @ \\
		    	\hline greater & \verb@ > @ \\
		    	\hline gt & \verb@ gt @ \\
		    	\hline lower\_equals & \verb@ <= @ \\
		    	\hline le & \verb@ le @ \\
		    	\hline greater\_equals & \verb@ >= @ \\
		    	\hline ge & \verb@ ge @ \\
		    	\hline comma & \verb@ , @ \\
		    	\hline concat\_mark & \verb@ \. @ \\
		    	\hline assign\_mark & \verb@ = @ \\
		    	\hline sub & \verb@ sub @ \\
		    	\hline if & \verb@ if @ \\
		    	\hline else & \verb@ else @ \\
		    	\hline elsif & \verb@ elsif @ \\
		    	\hline unless & \verb@ unless @ \\
		    	\hline return & \verb@ return @ \\
		    	\hline
		    \end{tabular}
		\end{center}
		
		
		    	%\hline defined & \verb@ defined @ \\
		    	%\hline int & \verb@ int @ \\
		    	%\hline length & \verb@ length @ \\
		    	%\hline print & \verb@ print @ \\
		    	%\hline scalar & \verb@ scalar @ \\
		    	%\hline substr & \verb@ substr @ \\
	
	\subsection{Remarques}
		La syntaxe complète de Perl concernant les noms de variables est beaucoup plus
		compliquée mais concerne des fonctionalités (packages) hors du cadre de ce
		projet, ce pourquoi nous nous sommes limités aux règles les plus simples.
	
\section{DFA}
	\subsection{Variables, comparateurs, blocs, litéraux}
		\begin{tikzpicture}[node distance=3cm,on grid,auto]
			\node[initial,initial text=,state] (0) {};
			\node[state] (6) [below=6cm of 0] {};
				\path[->] (0) edge node {'} (6);
				\path[->] (6) edge[loop right] node {[\^{}']} (6);
			\node[state] (1) [right of=6] {};
				\path[->] (0) edge node {\$} (1);
			\node[state,accepting] (11) [below of=1] {error};
				\path[->] (1) edge node{other} (11);
			\node[state] (5) [right of=1] {};
				\path[->] (1) edge node {[a-zA-Z\_]} (5);
				\path[->] (5) edge[loop above] node {[a-zA-Z0-9\_]} (5);
			\node[state,accepting] (12) [below of=5] {var};
				\path[->] (5) edge node{other} (12);
			\node[state,accepting] (7) [below of=6] {string};
				\path[->] (6) edge node{'} (7);
			\node[state] (3) [left of=6] {};
			\node[state] (2) [above of=3] {};
				\path[->] (0) edge node {[0-9]} (2);
				\path[->] (2) edge[loop right] node {[0-9]} (2);
			\node[state,accepting] (10) [left of=2] {int};
				\path[->] (2) edge node{other} (10);
				\path[->] (2) edge node {.} (3);
			\node[state,accepting] (8) [left of=3] {error};
				\path[->] (3) edge node{other} (8);
			\node[state] (4) [below of=3] {};
				\path[->] (3) edge node{[0-9]} (4);
				\path[->] (4) edge[loop below] node {[0-9]} (4);
			\node[state,accepting] (9) [left of=4] {float};
				\path[->] (4) edge node{other} (9);
			\node[state,accepting] (15) [left of=0] {ge};
			\node[state] (14) [above of=15] {};
				\path[->] (0) edge node{$>$} (14);
				\path[->] (14) edge node{=} (15);
			\node[state,accepting] (16) [left of=15] {gt};
				\path[->] (14) edge node{other} (16);
			\node[state] (13) [above of=14] {};
				\path[->] (0) edge node{$<$} (13);
			\node[state,accepting] (18) [left of=13] {lt};
				\path[->] (13) edge node{other} (18);
			\node[state,accepting] (17) [below of=18] {le};
				\path[->] (13) edge node{=} (17);
			\node[state] (19) [right of=13] {};
				\path[->] (0) edge node{=} (19);
			\node[state,accepting] (20) [above of=19] {eq};
				\path[->] (19) edge node{=} (20); 
			\node[state,accepting] (21) [left of=20] {assign};
				\path[->] (19) edge node[left]{other} (21);
			\node[state,accepting] (22) [right of=19] {lpar};
				\path[->] (0) edge node{(} (22);
			\node[state,accepting] (23) [right of=22] {rpar};
				\path[->] (0) edge node{)} (23);
			\node[state,accepting] (24) [below of=23] {lbrace};
				\path[->] (0) edge node{\{} (24);
			\node[state,accepting] (25) [below of=24] {rbrace};
				\path[->] (0) edge node{\}} (25);
			\node[state] (26) [below of=25] {};
				\path[->] (0) edge node{!} (26);
			\node[state,accepting] (27) [right of=26] {not};
				\path[->] (26) edge node{other} (27);
			\node[state,accepting] (28) [above of=27] {neq};
				\path[->] (26) edge node{=} (28);
		\end{tikzpicture}
		
	\subsection{Else, elsif et identifier}
		\begin{tikzpicture}[node distance=3cm,on grid,auto]
			\node[initial,initial text=,state] (0) {};
			\node[state] (1) [right of=0] {};
				\path[->] (0) edge node{e} (1);
			\node[state] (2) [right of=1] {};
				\path[->] (1) edge node{l} (2);
			\node[state,accepting] (11) [above of=2] {identifier};
				\path[->] (1) edge node{other} (11);
				\path[->] (2) edge node{other} (11);
			\node[state] (3) [right of=2] {};
				\path[->] (2) edge node{s} (3);
				\path[->] (3) edge[above right] node{other} (11);
			\node[state] (4) [right of=3] {};
				\path[->] (3) edge node{e} (4);
			\node[state,accepting] (5) [below of=4] {else};
				\path[->] (4) edge node{other} (5);
			\node[state] (12) [above of=4] {};
				\path[->] (4) edge node{[a-zA-Z0-9\_]} (12);
				\path[->] (12) edge[loop above] node{[a-zA-Z0-9\_]} (12);
				\path[->] (12) edge node{other} (11);
			\node[state] (6) [below=6cm of 3] {};
				\path[->] (3) edge node{i} (6);
			\node[state,accepting] (13) [right of=6] {identifier};
				\path[->] (6) edge node{other} (13);
			\node[state] (7) [below of=6] {};
				\path[->] (6) edge node{f} (7);
			\node[state,accepting] (8) [right of=7] {elsif};
				\path[->] (7) edge node{other} (8);
			\node[state] (9) [below=6cm of 1] {};
				\path[->] (1) edge[below left] node{[a-km-zA-Z0-9\_]} (9);
				\path[->] (2) edge[above] node{[a-rt-zA-Z0-9\_]} (9);
				\path[->] (3) edge[below right] node{[a-df-zA-Z0-9\_]} (9);
				\path[->] (6) edge[below] node{[a-eg-zA-Z0-9\_]} (9);
				\path[->] (7) edge[below] node{[a-zA-Z0-9\_]} (9);
				\path[->] (9) edge[loop below] node{[a-zA-Z0-9\_]} (9);
			\node[state,accepting] (10) [left of=9] {identifier};
				\path[->] (9) edge node{other} (10);
		\end{tikzpicture} \\
		Nous avons décidé de ne représenter que ces deux exemples, tous les mots clés fonctionnent sur le même principe.
		
	\subsection{Opérateurs et divers}
		\begin{tikzpicture}[node distance=3cm,on grid,auto]
			\node[initial,initial text=,state] (0) {};
			\node[state,accepting] (1) [below of=0] {plus};
				\path[->] (0) edge node{+} (1);
			\node[state,accepting] (2) [right of=1] {minus};
				\path[->] (0) edge node{-} (2);
			\node[state,accepting] (3) [above of=2] {times};
				\path[->] (0) edge node{*} (3);
			\node[state,accepting] (4) [above of=3] {divide};
				\path[->] (0) edge node{/} (4);
			\node[state,accepting] (5) [left of=4] {call};
				\path[->] (0) edge node{\&} (5);
			\node[state,accepting] (6) [left of=5] {semicolon};
				\path[->] (0) edge node{;} (6);	
			\node[state,accepting] (7) [left of=1] {comma};
				\path[->] (0) edge node{,} (7);	
			\node[state,accepting] (8) [left of=7] {concat\_mark};
				\path[->] (0) edge node{.} (8);	
		\end{tikzpicture}
		
	\subsection{Remarques}
		Certains tokens sont identifiables dès que leur dernier caractère a été lu (par exemple les accolades), 
		d'autres nécessitent la lecture du caractère suivant le dernier (par exemple, pour terminer un entier il faut lire autre chose qu'un chiffre).
		Dans ce deuxième cas, après avoir identifié le token la lecture du dernier caractère est annulée, il servira comme premier caractère du token suivant.

\section{Grammaire $LL_1$}

\newcounter{cnt}
\setcounter{cnt}{1}

\newcommand{\varname}[1]{\begin{math}\langle\end{math}#1\begin{math}\rangle\end{math}}
\newcommand{\num}{\begin{math}[\thecnt]\end{math}\addtocounter{cnt}{1}}

\subsection{Liste des règles}

\begin{longtable}{r l l}
	\centering
	\num & \varname{PROGRAM} & $\longrightarrow$ \varname{PROGRAM\_F} \varname{PROGRAM\_V} \\ [+10pt]
	\num & \varname{PROGRAM\_V} & $\longrightarrow$ \varname{PROGRAM\_F} \varname{PROGRAM\_V} \\
	\num & & $\longrightarrow$ $\epsilon$ \\ [+10pt]
	\num & \varname{PROGRAM\_F} & $\longrightarrow$ \varname{FUNCTION} \\
	\num & & $\longrightarrow$ \varname{INSTRUCTION} \\ [+10pt]
	\num & \varname{FUNCTION} & $\longrightarrow$ SUB IDENTIFIER \varname{FUNCTION\_ARGUMENT} \\ & & LBRACE \varname{INSTRUCTION\_LIST} RBRACE \\ [+10pt]
	\num & \varname{FUNCTION\_ARGUMENT} & $\longrightarrow$ LPAR \varname{ARGUMENT\_LIST} RPAR \\
	\num & & $\longrightarrow$ $\epsilon$ \\ [+10pt]
	\num & \varname{ARGUMENT\_LIST} & $\longrightarrow$ VAR \varname{ARGUMENT\_LIST\_V} \\
	\num & & $\longrightarrow$ $\epsilon$ \\ [+10pt]
	\num & \varname{ARGUMENT\_LIST\_V} & $\longrightarrow$ COMMA VAR \varname{ARGUMENT\_LIST\_V} \\
	\num & & $\longrightarrow$ $\epsilon$ \\ [+10pt]
	\num & \varname{INSTRUCTION\_LIST} & $\longrightarrow$ \varname{INSTRUCTION} \varname{INSTRUCTION\_LIST} \\
	\num & & $\longrightarrow$ $\epsilon$ \\ [+10pt]
	\num & \varname{INSTRUCTION} & $\longrightarrow$ \varname{EXPRESSION} \varname{INSTRUCTION\_F} SEMICOLON \\
	\num & & $\longrightarrow$ RETURN \varname{EXPRESSION} \varname{INSTRUCTION\_F} SEMICOLON \\
	\num & & $\longrightarrow$ LBRACE \varname{INSTRUCTION\_LIST} RBRACE \\
	\num & & $\longrightarrow$ \varname{CONDITION} \varname{EXPRESSION} LBRACE \varname{INSTRUCTION\_LIST} \\ & & RBRACE \varname{CONDITION\_END} \\ [+10pt]
	\num & \varname{INSTRUCTION\_F} & $\longrightarrow$ \varname{CONDITION} \varname{EXPRESSION} \\
	\num & & $\longrightarrow$ $\epsilon$ \\ [+10pt]
	\num & \varname{CONDITION} & $\longrightarrow$ IF \\
	\num & & $\longrightarrow$ UNLESS \\ [+10pt]
	\num & \varname{CONDITION\_END} & $\longrightarrow$ ELSIF \varname{EXPRESSION} LBRACE \varname{INSTRUCTION\_LIST} \\ & & RBRACE \varname{CONDITION\_END} \\
	\num & & $\longrightarrow$ ELSE LBRACE \varname{INSTRUCTION\_LIST} RBRACE \\
	\num & & $\longrightarrow$ $\epsilon$ \\ [+10pt]
	\num & \varname{EXPRESSION} & $\longrightarrow$ \varname{EXPRESSION\_TWO} \varname{EXPRESSION\_V} \\ [+10pt]
	\num & \varname{EXPRESSION\_V} & $\longrightarrow$ ASSIGN\_MARK \varname{EXPRESSION\_TWO} \varname{EXPRESSION\_V} \\
	\num & & $\longrightarrow$ $\epsilon$ \\ [+10pt]
	\num & \varname{EXPRESSION\_TWO} & $\longrightarrow$ \varname{EXPRESSION\_THREE} \varname{EXPRESSION\_TWO\_V} \\ [+10pt]
	\num & \varname{EXPRESSION\_TWO\_V} & $\longrightarrow$ LAZY\_OR \varname{EXPRESSION\_THREE} \varname{EXPRESSION\_TWO\_V} \\
	\num & & $\longrightarrow$ $\epsilon$ \\ [+10pt]
	\num & \varname{EXPRESSION\_THREE} & $\longrightarrow$ \varname{EXPRESSION\_FOUR} \varname{EXPRESSION\_THREE\_V} \\ [+10pt]
	\num & \varname{EXPRESSION\_THREE\_V} & $\longrightarrow$ LAZY\_AND \varname{EXPRESSION\_FOUR} \varname{EXPRESSION\_THREE\_V} \\
	\num & & $\longrightarrow$ $\epsilon$ \\ [+10pt]
	\num & \varname{EXPRESSION\_FOUR} & $\longrightarrow$ \varname{EXPRESSION\_FIVE} \varname{EXPRESSION\_FOUR\_V} \\ [+10pt]
	\num & \varname{EXPRESSION\_FOUR\_V} & $\longrightarrow$ \varname{EXPRESSION\_FOUR\_F} \varname{EXPRESSION\_FIVE} \\
	\num & & $\longrightarrow$ $\epsilon$ \\ [+10pt]
	\num & \varname{EXPRESSION\_FOUR\_F} & $\longrightarrow$ DIFFERENT \\
	\num & & $\longrightarrow$ EQ \\
	\num & & $\longrightarrow$ EQUALS \\
	\num & & $\longrightarrow$ NE \\ [+10pt]
	\num & \varname{EXPRESSION\_FIVE} & $\longrightarrow$ \varname{EXPRESSION\_SIX} \varname{EXPRESSION\_FIVE\_V} \\ [+10pt]
	\num & \varname{EXPRESSION\_FIVE\_V} & $\longrightarrow$ \varname{EXPRESSION\_FIVE\_F} \varname{EXPRESSION\_SIX} \\
	\num & & $\longrightarrow$ $\epsilon$ \\ [+10pt]
	\num & \varname{EXPRESSION\_FIVE\_F} & $\longrightarrow$ GREATER \\
	\num & & $\longrightarrow$ GREATER\_EQUALS \\
	\num & & $\longrightarrow$ GT \\
	\num & & $\longrightarrow$ GE \\
	\num & & $\longrightarrow$ LOWER \\
	\num & & $\longrightarrow$ LOWER\_EQUALS \\
	\num & & $\longrightarrow$ LT \\
	\num & & $\longrightarrow$ LE \\ [+10pt]
	\num & \varname{EXPRESSION\_SIX} & $\longrightarrow$ \varname{EXPRESSION\_SEVEN} \varname{EXPRESSION\_SIX\_V} \\ [+10pt]
	\num & \varname{EXPRESSION\_SIX\_V} & $\longrightarrow$ \varname{EXPRESSION\_SIX\_F} \varname{EXPRESSION\_SEVEN} \\ & & \varname{EXPRESSION\_SIX\_V} \\
	\num & & $\longrightarrow$ $\epsilon$ \\ [+10pt]
	\num & \varname{EXPRESSION\_SIX\_F} & $\longrightarrow$ PLUS \\
	\num & & $\longrightarrow$ MINUS \\
	\num & & $\longrightarrow$ CONCAT\_MARK \\ [+10pt]
	\num & \varname{EXPRESSION\_SEVEN} & $\longrightarrow$ \varname{EXPRESSION\_EIGHT} \varname{EXPRESSION\_SEVEN\_V} \\ [+10pt]
	\num & \varname{EXPRESSION\_SEVEN\_V} & $\longrightarrow$ \varname{EXPRESSION\_SEVEN\_F} \varname{EXPRESSION\_EIGHT} \\ & & \varname{EXPRESSION\_SEVEN\_V} \\
	\num & & $\longrightarrow$ $\epsilon$ \\ [+10pt]
	\num & \varname{EXPRESSION\_SEVEN\_F} & $\longrightarrow$ TIMES \\
	\num & & $\longrightarrow$ DIVIDE \\ [+10pt]
	\num & \varname{EXPRESSION\_EIGHT} & $\longrightarrow$ \varname{EXPRESSION\_NINE} \\
	\num & & $\longrightarrow$ \varname{EXPRESSION\_EIGHT\_F} \varname{EXPRESSION\_EIGHT} \\ [+10pt]
	\num & \varname{EXPRESSION\_EIGHT\_F} & $\longrightarrow$ NOT  \\
	\num & & $\longrightarrow$ PLUS \\
	\num & & $\longrightarrow$ MINUS \\ [+10pt]
	\num & \varname{EXPRESSION\_NINE} & $\longrightarrow$ LPAR \varname{EXPRESSION} RPAR \\
	\num & & $\longrightarrow$ \varname{SIMPLE\_EXPRESSION} \\ [+10pt]
	\num & \varname{SIMPLE\_EXPRESSION} & $\longrightarrow$ \varname{FUNCTION\_CALL} \\
	\num & & $\longrightarrow$ INTEGER \\
	\num & & $\longrightarrow$ FLOAT \\
	\num & & $\longrightarrow$ STRING \\
	\num & & $\longrightarrow$ VAR \\ [+10pt]
	\num & \varname{FUNCTION\_CALL} & $\longrightarrow$ CALL\_MARK IDENTIFIER LPAR \varname{ARGUMENT\_CALL\_LIST} \\ & & RPAR \\ [+10pt]
	\num & \varname{ARGUMENT\_CALL\_LIST} & $\longrightarrow$ \varname{EXPRESSION} \varname{ARGUMENT\_CALL\_LIST\_V} \\
	\num & & $\longrightarrow$ $\epsilon$ \\ [+10pt]
	\num & \varname{ARGUMENT\_CALL\_LIST\_V} & $\longrightarrow$ COMMA \varname{EXPRESSION} \varname{ARGUMENT\_CALL\_LIST\_V} \\
	\num & & $\longrightarrow$ $\epsilon$ \\
\end{longtable}

\newcommand{\eps}{$\epsilon$}

\pagebreak

	\subsection{$First_1$}
		\small
		\begin{longtable}{l l l}
		\varname{PROGRAM}					& : & SUB, RETURN, LBRACE, LPAR, INTEGER, FLOAT, STRING, VAR \\ & & , CALL\_MARK, IF, UNLESS, NOT, PLUS, MINUS \\ [+10pt]
		\varname{PROGRAM\_V}				& : & SUB, RETURN, LBRACE, LPAR, INTEGER, FLOAT, STRING, VAR \\ & & , CALL\_MARK, IF, UNLESS, NOT, PLUS, MINUS, \eps \\ [+10pt]
		\varname{PROGRAM\_F}				& : & SUB, RETURN, LBRACE, LPAR, INTEGER, FLOAT, STRING, VAR \\ & & , CALL\_MARK, IF, UNLESS, NOT, PLUS, MINUS \\ [+10pt]
		\varname{FUNCTION}					& : & SUB \\ [+10pt]
		\varname{FUNCTION\_ARGUMENT}		& : & LPAR, \eps \\ [+10pt]
		\varname{ARGUMENT\_LIST}			& : & VAR, \eps \\ [+10pt]
		\varname{ARGUMENT\_LIST\_V}			& : & COMMA, \eps \\ [+10pt]
		\varname{INSTRUCTION\_LIST}			& : & RETURN, LBRACE, LPAR, INTEGER, FLOAT, STRING, VAR \\ & & , CALL\_MARK, IF, UNLESS, NOT, PLUS, MINUS, \eps \\ [+10pt]
		\varname{FUNCTION\_CALL}			& : & CALL\_MARK \\ [+10pt]
		\varname{ARGUMENT\_CALL\_LIST}		& : & LPAR, INTEGER, FLOAT, STRING, VAR, CALL\_MARK, NOT, PLUS, MINUS, \eps \\ [+10pt]
		\varname{ARGUMENT\_CALL\_LIST\_V}	& : & COMMA, \eps \\ [+10pt]
		\varname{INSTRUCTION}				& : & RETURN, LBRACE, LPAR, INTEGER, FLOAT, STRING, VAR \\ & & , CALL\_MARK, IF, UNLESS, NOT, PLUS, MINUS \\ [+10pt]
		\varname{INSTRUCTION\_F}			& : & IF, UNLESS, \eps \\ [+10pt]
		\varname{CONDITION}					& : & IF, UNLESS \\ [+10pt]
		\varname{CONDITION\_END}			& : & ELSIF, ELSE, \eps \\ [+10pt]
		\varname{EXPRESSION}				& : & LPAR, INTEGER, FLOAT, STRING, VAR, CALL\_MARK, NOT, PLUS, MINUS \\ [+10pt]
		\varname{EXPRESSION\_V}				& : & ASSIGN\_MARK, \eps \\ [+10pt]
		\varname{EXPRESSION\_TWO}			& : & LPAR, INTEGER, FLOAT, STRING, VAR, CALL\_MARK, NOT, PLUS, MINUS \\ [+10pt]
		\varname{EXPRESSION\_TWO\_V}		& : & LAZY\_OR, \eps \\ [+10pt]
		\varname{EXPRESSION\_THREE}			& : & LPAR, INTEGER, FLOAT, STRING, VAR, CALL\_MARK, NOT, PLUS, MINUS \\ [+10pt]
		\varname{EXPRESSION\_THREE\_V}		& : & LAZY\_AND, \eps \\ [+10pt]
		\varname{EXPRESSION\_FOUR}			& : & LPAR, INTEGER, FLOAT, STRING, VAR, CALL\_MARK, NOT, PLUS, MINUS \\ [+10pt]
		\varname{EXPRESSION\_FOUR\_V}		& : & DIFFERENT, EQ, EQUALS, NE, \eps \\ [+10pt]
		\varname{EXPRESSION\_FOUR\_F}		& : & DIFFERENT, EQ, EQUALS, NE \\ [+10pt]
		\varname{EXPRESSION\_FIVE}			& : & LPAR, INTEGER, FLOAT, STRING, VAR, CALL\_MARK, NOT, PLUS, MINUS \\ [+10pt]
		\varname{EXPRESSION\_FIVE\_V}		& : & GE, GREATER, GREATER\_EQUALS, GT, LE, LOWER, LOWER\_EQUALS, LT, \eps \\ [+10pt]
		\varname{EXPRESSION\_FIVE\_F}		& : & GE, GREATER, GREATER\_EQUALS, GT, LE, LOWER, LOWER\_EQUALS, LT \\ [+10pt]
		\varname{EXPRESSION\_SIX}			& : & LPAR, INTEGER, FLOAT, STRING, VAR, CALL\_MARK, NOT, PLUS, MINUS \\ [+10pt]
		\varname{EXPRESSION\_SIX\_V}		& : & PLUS, MINUS, CONCAT\_MARK, \eps \\ [+10pt]
		\varname{EXPRESSION\_SIX\_F}		& : & PLUS, MINUS, CONCAT\_MARK \\ [+10pt]
		\varname{EXPRESSION\_SEVEN}			& : & LPAR, INTEGER, FLOAT, STRING, VAR, CALL\_MARK, NOT, PLUS, MINUS \\ [+10pt]
		\varname{EXPRESSION\_SEVEN\_V}		& : & TIMES, DIVIDE, \eps \\ [+10pt]
		\varname{EXPRESSION\_SEVEN\_F}		& : & TIMES, DIVIDE \\ [+10pt]
		\varname{EXPRESSION\_EIGHT}			& : & LPAR, INTEGER, FLOAT, STRING, VAR, CALL\_MARK, NOT, PLUS, MINUS \\ [+10pt]
		\varname{EXPRESSION\_EIGHT\_F}		& : & NOT, PLUS, MINUS \\ [+10pt]
		\varname{EXPRESSION\_NINE}			& : & LPAR, INTEGER, FLOAT, STRING, VAR, CALL\_MARK \\ [+10pt]
		\varname{SIMPLE\_EXPRESSION}		& : & INTEGER, FLOAT, STRING, VAR, CALL\_MARK \\ [+10pt]
		\end{longtable}
		
	\subsection{$Follow_1$}
		\small
		\begin{longtable}{l l l}
		\varname{PROGRAM}					& : & $\phi$ \\ [+10pt]
		\varname{PROGRAM\_V}				& : & $\phi$ \\ [+10pt]
		\varname{PROGRAM\_F}				& : & SUB, RETURN, LBRACE, LPAR, INTEGER, FLOAT, STRING, VAR, \\ & & CALL\_MARK, IF, UNLESS, NOT, PLUS, MINUS \\ [+10pt]
		\varname{FUNCTION}					& : & SUB, RETURN, LBRACE, LPAR, INTEGER, FLOAT, STRING, VAR, \\ & & CALL\_MARK, IF, UNLESS, NOT, PLUS, MINUS \\ [+10pt]
		\varname{FUNCTION\_ARGUMENT}		& : & LBRACE \\ [+10pt]
		\varname{ARGUMENT\_LIST}			& : & RPAR \\ [+10pt]
		\varname{ARGUMENT\_LIST\_V}			& : & RPAR \\ [+10pt]
		\varname{INSTRUCTION\_LIST}			& : & RBRACE \\ [+10pt]
		\varname{FUNCTION\_CALL}			& : & IF, UNLESS, COMMA, LBRACE, RPAR, SEMICOLON, ASSIGN\_MARK, \\ & & LAZY\_OR, LAZY\_AND, DIFFERENT, EQ, EQUALS, NE, GE, GREATER, \\ & & GREATER\_EQUALS, GT, LE, LOWER, LOWER\_EQUALS, LT, \\ & & PLUS, MINUS, CONCAT\_MARK, TIMES, DIVIDE \\ [+10pt]
		\varname{ARGUMENT\_CALL\_LIST}		& : & RPAR \\ [+10pt]
		\varname{ARGUMENT\_CALL\_LIST\_V}	& : & RPAR \\ [+10pt]
		\varname{INSTRUCTION}				& : & RBRACE, SUB, RETURN, LBRACE, LPAR, INTEGER, FLOAT, STRING, VAR, \\ & & CALL\_MARK, IF, UNLESS, NOT, PLUS, MINUS \\ [+10pt]
		\varname{INSTRUCTION\_F}			& : & SEMICOLON \\ [+10pt]
		\varname{CONDITION}					& : & LPAR, INTEGER, FLOAT, STRING, VAR, CALL\_MARK,  NOT, PLUS, MINUS \\ [+10pt]
		\varname{CONDITION\_END}			& : & RBRACE, SUB, RETURN, LBRACE, LPAR, INTEGER, FLOAT, STRING, VAR, \\ & & CALL\_MARK, IF, UNLESS, NOT, PLUS, MINUS \\ [+10pt]
		\varname{EXPRESSION}				& : & IF, UNLESS, COMMA, LBRACE, RPAR, SEMICOLON \\ [+10pt]
		\varname{EXPRESSION\_V}				& : & IF, UNLESS, COMMA, LBRACE, RPAR, SEMICOLON \\ [+10pt]
		\varname{EXPRESSION\_TWO}			& : & IF, UNLESS, COMMA, LBRACE, RPAR, SEMICOLON, ASSIGN\_MARK \\ [+10pt]
		\varname{EXPRESSION\_TWO\_V}		& : & IF, UNLESS, COMMA, LBRACE, RPAR, SEMICOLON, ASSIGN\_MARK \\ [+10pt]
		\varname{EXPRESSION\_THREE}			& : & IF, UNLESS, COMMA, LBRACE, RPAR, SEMICOLON, ASSIGN\_MARK, \\ & & LAZY\_OR, LAZY\_AND \\ [+10pt]
		\varname{EXPRESSION\_THREE\_V}		& : & IF, UNLESS, COMMA, LBRACE, RPAR, SEMICOLON, ASSIGN\_MARK, \\ & & LAZY\_OR, LAZY\_AND \\ [+10pt]
		\varname{EXPRESSION\_FOUR}			& : & IF, UNLESS, COMMA, LBRACE, RPAR, SEMICOLON, ASSIGN\_MARK, \\ & & LAZY\_OR, LAZY\_AND, DIFFERENT, EQ, EQUALS, NE \\ [+10pt]
		\varname{EXPRESSION\_FOUR\_V}		& : & IF, UNLESS, COMMA, LBRACE, RPAR, SEMICOLON, ASSIGN\_MARK, \\ & & LAZY\_OR, LAZY\_AND, DIFFERENT, EQ, EQUALS, NE \\ [+10pt]
		\varname{EXPRESSION\_FOUR\_F}		& : & LPAR, INTEGER, FLOAT, STRING, VAR, CALL\_MARK, NOT, PLUS, MINUS \\ [+10pt]
		\varname{EXPRESSION\_FIVE}			& : & IF, UNLESS, COMMA, LBRACE, RPAR, SEMICOLON, ASSIGN\_MARK, \\ & & LAZY\_OR, LAZY\_AND, DIFFERENT, EQ, EQUALS, NE, GE, GREATER, \\ & & GREATER\_EQUALS, GT, LE, LOWER, LOWER\_EQUALS, LT \\ [+10pt]
		\varname{EXPRESSION\_FIVE\_V}		& : & IF, UNLESS, COMMA, LBRACE, RPAR, SEMICOLON, ASSIGN\_MARK, \\ & & LAZY\_OR, LAZY\_AND, DIFFERENT, EQ, EQUALS, NE, GE, GREATER, \\ & & GREATER\_EQUALS, GT, LE, LOWER, LOWER\_EQUALS, LT \\ [+10pt]
		\varname{EXPRESSION\_FIVE\_F}		& : & LPAR, INTEGER, FLOAT, STRING, VAR, CALL\_MARK, NOT, PLUS, MINUS \\ [+10pt]
		\varname{EXPRESSION\_SIX}			& : & IF, UNLESS, COMMA, LBRACE, RPAR, SEMICOLON, ASSIGN\_MARK, \\ & & LAZY\_OR, LAZY\_AND, DIFFERENT, EQ, EQUALS, NE, GE, GREATER, \\ & & GREATER\_EQUALS, GT, LE, LOWER, LOWER\_EQUALS, LT, \\ & & PLUS, MINUS, CONCAT\_MARK \\ [+10pt]
		\varname{EXPRESSION\_SIX\_V}		& : & IF, UNLESS, COMMA, LBRACE, RPAR, SEMICOLON, ASSIGN\_MARK, \\ & & LAZY\_OR, LAZY\_AND, DIFFERENT, EQ, EQUALS, NE, GE, GREATER, \\ & & GREATER\_EQUALS, GT, LE, LOWER, LOWER\_EQUALS, LT, \\ & & PLUS, MINUS, CONCAT\_MARK \\ [+10pt]
		\varname{EXPRESSION\_SIX\_F}		& : & LPAR, INTEGER, FLOAT, STRING, VAR, CALL\_MARK, NOT, PLUS, MINUS \\ [+10pt]
		\varname{EXPRESSION\_SEVEN}			& : & IF, UNLESS, COMMA, LBRACE, RPAR, SEMICOLON, ASSIGN\_MARK, \\ & & LAZY\_OR, LAZY\_AND, DIFFERENT, EQ, EQUALS, NE, GE, GREATER, \\ & & GREATER\_EQUALS, GT, LE, LOWER, LOWER\_EQUALS, LT, \\ & & PLUS, MINUS, CONCAT\_MARK, TIMES, DIVIDE \\ [+10pt]
		\varname{EXPRESSION\_SEVEN\_V}		& : & IF, UNLESS, COMMA, LBRACE, RPAR, SEMICOLON, ASSIGN\_MARK, \\ & & LAZY\_OR, LAZY\_AND, DIFFERENT, EQ, EQUALS, NE, GE, GREATER, \\ & & GREATER\_EQUALS, GT, LE, LOWER, LOWER\_EQUALS, LT, \\ & & PLUS, MINUS, CONCAT\_MARK, TIMES, DIVIDE \\ [+10pt]
		\varname{EXPRESSION\_SEVEN\_F}		& : & LPAR, INTEGER, FLOAT, STRING, VAR, CALL\_MARK, NOT, PLUS, MINUS \\ [+10pt]
		\varname{EXPRESSION\_EIGHT}			& : & IF, UNLESS, COMMA, LBRACE, RPAR, SEMICOLON, ASSIGN\_MARK, \\ & & LAZY\_OR, LAZY\_AND, DIFFERENT, EQ, EQUALS, NE, GE, GREATER, \\ & & GREATER\_EQUALS, GT, LE, LOWER, LOWER\_EQUALS, LT, \\ & & PLUS, MINUS, CONCAT\_MARK, TIMES, DIVIDE \\ [+10pt]
		\varname{EXPRESSION\_EIGHT\_F}		& : & LPAR, INTEGER, FLOAT, STRING, VAR, CALL\_MARK, NOT, PLUS, MINUS \\ [+10pt]
		\varname{EXPRESSION\_NINE}			& : & IF, UNLESS, COMMA, LBRACE, RPAR, SEMICOLON, ASSIGN\_MARK, \\ & & LAZY\_OR, LAZY\_AND, DIFFERENT, EQ, EQUALS, NE, GE, GREATER, \\ & & GREATER\_EQUALS, GT, LE, LOWER, LOWER\_EQUALS, LT, \\ & & PLUS, MINUS, CONCAT\_MARK, TIMES, DIVIDE \\ [+10pt]
		\varname{SIMPLE\_EXPRESSION}		& : & IF, UNLESS, COMMA, LBRACE, RPAR, SEMICOLON, ASSIGN\_MARK, \\ & & LAZY\_OR, LAZY\_AND, DIFFERENT, EQ, EQUALS, NE, GE, GREATER, \\ & & GREATER\_EQUALS, GT, LE, LOWER, LOWER\_EQUALS, LT, \\ & & PLUS, MINUS, CONCAT\_MARK, TIMES, DIVIDE \\ [+10pt]
		\end{longtable}	

\section{Table d'actions}

\newcommand{\g}{$>$}
\newcommand{\h}{$<$}
\newcommand{\amp}{$\&$}
\newcommand{\ddamp}{$\&\&$}
\newcommand{\vertl}{$\|$}
\begin{landscape}
\tabcolsep=0.10cm
\small
\begin{longtable}{|c|c|c|c|c|c|c|c|c|c|c|c|c|c|c|c|c|c|c|c|c|c|c|c|c|c|c|c|c|}
\hline									&if	&unless	&else	&elsif 	&sub&return	&=	&\{	&\}	&(	&) 	&,	&;	&\amp 	&\vertl	&\ddamp	&!=	&eq &== &ne &\g	&\g=&gt &ge &\h	&\h=&lt &le \\
\hline	\varname{PROGRAM}				&1 	&1 		& 		& 		&1	&		&	&1	& 	&1	&1	& 	&	&1		& 		& 		& 	&	&	&	&	&	&	&	&	&	&	&	\\ 
\hline	\varname{PROGRAM\_V} 			&2 	&2 		& 		& 		&2	&		&	&2	& 	&2	&2	& 	&	&2		& 		& 		& 	&	&	&	&	&	&	&	&	&	&	&	\\ 
\hline	\varname{PROGRAM\_F}			&5	&5 		& 		& 		&4	&		&	&5 	&5 	&5	&5 	& 	&	&5		& 		& 		& 	&	&	&	&	&	&	&	&	&	&	&	\\ 
\hline	\varname{FUNCTION} 				&	& 		& 		& 		&6	&		&	& 	& 	&	& 	& 	&	&		& 		& 		& 	&	&	&	&	&	&	&	&	&	&	&	\\ 
\hline	\varname{FUNCTION\_ARGUMENT} 	&	& 		& 		& 		&	&		&	&8 	& 	&7	& 	& 	&	&		& 		& 		& 	&	&	&	&	&	&	&	&	&	&	&	\\ 
\hline	\varname{ARGUMENT\_LIST} 		&	& 		& 		& 		&	&		&	& 	& 	&	&10	& 	&	&		& 		& 		& 	&	&	&	&	&	&	&	&	&	&	&	\\ 
\hline	\varname{ARGUMENT\_LIST\_V} 	&	& 		& 		& 		&	&		&	& 	& 	&	&12	&11	&	&		& 		& 		& 	&	&	&	&	&	&	&	&	&	&	&	\\ 
\hline	\varname{INSTRUCTION\_LIST} 	&13	&13		& 		& 		&	&13		&	&13	&14	&13	&	&	&	&13		& 		& 		& 	&	&	&	&	&	&	&	&	&	&	&	\\ 
\hline	\varname{INSTRUCTION} 			&18	&18		& 		& 		&	&16		&	&17	&	&15	&	&	&	&15		& 		& 		& 	&	&	&	&	&	&	&	&	&	&	&	\\ 
\hline	\varname{INSTRUCTION\_F} 		&19	&19		& 		& 		&	&		&	&	&	&	&	&	&20	&		& 		& 		& 	&	&	&	&	&	&	&	&	&	&	&	\\ 
\hline	\varname{CONDITION} 			&21	&22		& 		& 		&	&		&	&	&	&	&	&	&	&		& 		& 		& 	&	&	&	&	&	&	&	&	&	&	&	\\ 
\hline	\varname{CONDITION\_END} 		&	&		&24 	&23		&25	&25		&	&25	&25	&25	&	&	&	&25		& 		& 		& 	&	&	&	&	&	&	&	&	&	&	&	\\ 
\hline	\varname{SIMPLE\_EXPRESSION} 	&	&		&	 	&		&	&		&	&	&	&	&	&	&	&71		&		&		&	&	&	&	&	&	&	&	&	&	&	&	\\ 
\hline	\varname{FUNCTION\_CALL} 		&	&		&	 	&		&	&		&	&	&	&	&	&	&	&76		&		&		&	&	&	&	&	&	&	&	&	&	&	&	\\ 
\hline	\varname{ARGUMENT\_CALL\_LIST} 	&	& 		& 		& 		&	&		&	& 	& 	&77	&78	& 	&	&77		& 		& 		& 	&	&	&	&	&	&	&	&	&	&	&	\\ 
\hline	\varname{ARGUMENT\_CALL\_LIST\_V} &	& 		& 		& 		&	&		&	& 	& 	&	&80	&79	&	&		& 		& 		& 	&	&	&	&	&	&	&	&	&	&	&	\\ 
\hline
\end{longtable}
\end{landscape}

\begin{landscape}
\tabcolsep=0.10cm
\small
\begin{longtable}{|c|c|c|c|c|c|c|c|c|c|c|c|c|c|c|c|c|c|c|c|c|c|c|c|c|c|c|c|c|}
\hline									&if	&unless	&else	&elsif 	&sub&return	&=	&\{	&\}	&(	&) 	&,	&;	&\amp 	&\vertl	&\ddamp &!=	&eq &== &ne &\g	&\g=&gt &ge &\h	&\h=&lt &le \\
\hline	\varname{EXPRESSION} 			&	&		&	 	&		&	&		&	&	&	&26	&	&	&	&26		& 		& 		& 	&	&	&	&	&	&	&	&	&	&	&	\\
\hline	\varname{EXPRESSION\_V} 		&28	&28		&	 	&		&	&		&27	&28	&	&	&28	&28	&28	&		& 		& 		& 	&	&	&	&	&	&	&	&	&	&	&	\\
\hline	\varname{EXPRESSION\_TWO} 		&	&		&	 	&		&	&		&	&	&	&29	&	&	&	&29		& 		& 		& 	&	&	&	&	&	&	&	&	&	&	&	\\
\hline	\varname{EXPRESSION\_TWO\_V} 	&31	&31		&	 	&		&	&		&31	&31	&	&	&31	&31	&31	&		&30		& 		& 	&	&	&	&	&	&	&	&	&	&	&	\\
\hline	\varname{EXPRESSION\_THREE} 	&	&		&	 	&		&	&		&	&	&	&32	&	&	&	&32		& 		& 		& 	&	&	&	&	&	&	&	&	&	&	&	\\
\hline	\varname{EXPRESSION\_THREE\_V} 	&34	&34		&	 	&		&	&		&34	&34	&	&	&34	&34	&34	&		&34		&33		& 	&	&	&	&	&	&	&	&	&	&	&	\\
\hline	\varname{EXPRESSION\_FOUR} 		&	&		&	 	&		&	&		&	&	&	&35	&	&	&	&35		& 		& 		& 	&	&	&	&	&	&	&	&	&	&	&	\\
\hline	\varname{EXPRESSION\_FOUR\_V} 	&37	&37		&	 	&		&	&		&37	&37	&	&	&37	&37	&37	&		&37		&37		&36	&36	&36	&36	&	&	&	&	&	&	&	&	\\
\hline	\varname{EXPRESSION\_FOUR\_F} 	&	&		&	 	&		&	&		&	&	&	&	&	&	&	&		&		&		&38	&39	&40	&41	&	&	&	&	&	&	&	&	\\
\hline	\varname{EXPRESSION\_FIVE} 		&	&		&	 	&		&	&		&	&	&	&42	&	&	&	&42		& 		& 		& 	&	&	&	&	&	&	&	&	&	&	&	\\
\hline	\varname{EXPRESSION\_FIVE\_V} 	&44	&44		&	 	&		&	&		&44	&44	&	&	&44	&44	&44	&		&44		&44		&44	&44	&44	&44	&43	&43	&43	&43	&43	&43	&43	&43	\\
\hline	\varname{EXPRESSION\_FIVE\_F} 	&	&		&	 	&		&	&		&	&	&	&	&	&	&	&		&		&		&	&	&	&	&45	&46	&47	&48	&49	&50	&51	&52	\\
\hline	\varname{EXPRESSION\_SIX} 		&	&		&	 	&		&	&		&	&	&	&53	&	&	&	&53		& 		& 		& 	&	&	&	&	&	&	&	&	&	&	&	\\
\hline	\varname{EXPRESSION\_SIX\_V} 	&55	&55		&	 	&		&	&		&55	&55	&	&	&55	&55	&55	&		&55		&55		&55	&55	&55	&55	&55	&55	&55	&55	&55	&55	&55	&55	\\
\hline	\varname{EXPRESSION\_SIX\_F} 	&	&		&	 	&		&	&		&	&	&	&	&	&	&	&		&		&		&	&	&	&	&	&	&	&	&	&	&	&	\\
\hline	\varname{EXPRESSION\_SEVEN} 	&	&		&	 	&		&	&		&	&	&	&59	&	&	&	&59		& 		& 		& 	&	&	&	&	&	&	&	&	&	&	&	\\
\hline	\varname{EXPRESSION\_SEVEN\_V} 	&61	&61		&	 	&		&	&		&61	&61	&	&	&61	&61	&61	&		&61		&61		&61	&61	&61	&61	&61	&61	&61	&61	&61	&61	&61	&61	\\
\hline	\varname{EXPRESSION\_SEVEN\_F} 	&	&		&	 	&		&	&		&	&	&	&	&	&	&	&		&		&		&	&	&	&	&	&	&	&	&	&	&	&	\\
\hline	\varname{EXPRESSION\_EIGHT} 	&	&		&	 	&		&	&		&	&	&	&64	&	&	&	&64		&		&		&	&	&	&	&	&	&	&	&	&	&	&	\\
\hline	\varname{EXPRESSION\_EIGHT\_F} 	&	&		&	 	&		&	&		&	&	&	&	&	&	&	&		&		&		&	&	&	&	&	&	&	&	&	&	&	&	\\
\hline	\varname{EXPRESSION\_NINE} 		&	&		&	 	&		&	&		&	&	&	&69	&	&	&	&70		&		&		&	&	&	&	&	&	&	&	&	&	&	&	\\
\hline
\end{longtable}
\end{landscape}

\begin{landscape}
\tabcolsep=0.10cm
\small
\begin{longtable}{|c|c|c|c|c|c|c|c|c|c|c|c|}			
\hline									&+ 	&- 	&. 	&* 	&/ 	&! 	&id &integer&float	&string	&var\\
\hline	\varname{PROGRAM}				&1	&1	& 	& 	&	&1	& 	&1		&1		&1		&1	\\
\hline	\varname{PROGRAM\_V} 			&2	&2	& 	& 	&	&2	& 	&2		&2		&2		&2	\\
\hline	\varname{PROGRAM\_F}            &5 	&5 	& 	& 	&	&5 	& 	&5 		&5 		&5 		&5 	\\
\hline	\varname{FUNCTION} 				& 	& 	& 	& 	&	& 	& 	& 		& 		& 		& 	\\
\hline	\varname{FUNCTION\_ARGUMENT} 	& 	& 	& 	& 	&	& 	& 	& 		& 		& 		& 	\\
\hline	\varname{ARGUMENT\_LIST} 		& 	& 	& 	& 	&	& 	& 	& 		& 		& 		&9 	\\
\hline	\varname{ARGUMENT\_LIST\_V} 	& 	& 	& 	& 	&	& 	& 	& 		& 		& 		&	\\
\hline	\varname{INSTRUCTION\_LIST} 	&13	&13	& 	& 	&	&13	& 	&13		&13		&13		&13	\\
\hline	\varname{INSTRUCTION} 			&15	&15	& 	& 	&	&15	& 	&15		&15		&15		&15	\\
\hline	\varname{INSTRUCTION\_F} 		&	&	& 	& 	&	&	& 	&		&		&		&	\\
\hline	\varname{CONDITION} 			&	&	& 	& 	&	&	& 	&		&		&		&	\\
\hline	\varname{CONDITION\_END} 		&25	&25	& 	& 	&	&25	& 	&25		&25		&25		&25	\\
\hline	\varname{SIMPLE\_EXPRESSION} 	&	&	&	&	&	&	& 	&72		&73		&74		&75	\\
\hline	\varname{FUNCTION\_CALL} 		&	&	&	&	&	&	& 	&		&		&		&	\\
\hline	\varname{ARGUMENT\_CALL\_LIST} 	&77 &77	& 	& 	&	&77	& 	&77		&77		&77		&77	\\
\hline	\varname{ARGUMENT\_CALL\_LIST\_V}& 	& 	& 	& 	&	& 	& 	& 		& 		& 		&	\\
\hline
\end{longtable}                 
\end{landscape}


\begin{landscape}
\begin{longtable}{|c|c|c|c|c|c|c|c|c|c|c|c|}				
\hline									&+ 	&- 	&. 	&* 	&/ 	&! 	&id &integer&float	&string	&var\\
\hline	\varname{EXPRESSION} 			&26	&26	& 	& 	&	&26	& 	&26		&26		&26		&26	\\
\hline	\varname{EXPRESSION\_V} 		&	&	& 	& 	&	&	& 	&		&		&		&	\\
\hline	\varname{EXPRESSION\_TWO} 		&29	&29	& 	& 	&	&29	& 	&29		&29		&29		&29	\\
\hline	\varname{EXPRESSION\_TWO\_V} 	&	&	& 	& 	&	&	& 	&		&		&		&	\\
\hline	\varname{EXPRESSION\_THREE} 	&32	&32	& 	& 	&	&32	& 	&32		&32		&32		&32	\\
\hline	\varname{EXPRESSION\_THREE\_V} 	&	&	& 	& 	&	&	& 	&		&		&		&	\\
\hline	\varname{EXPRESSION\_FOUR} 		&35	&35	& 	& 	&	&35	& 	&35		&35		&35		&35	\\
\hline	\varname{EXPRESSION\_FOUR\_V} 	&	&	& 	& 	&	&	& 	&		&		&		&	\\
\hline	\varname{EXPRESSION\_FOUR\_F} 	&	&	& 	& 	&	&	& 	&		&		&		&	\\
\hline	\varname{EXPRESSION\_FIVE} 		&42	&42	& 	& 	&	&42	& 	&42		&42		&42		&42	\\
\hline	\varname{EXPRESSION\_FIVE\_V} 	&	&	& 	& 	&	&	& 	&		&		&		&	\\
\hline	\varname{EXPRESSION\_FIVE\_F} 	&	&	& 	& 	&	&	& 	&		&		&		&	\\
\hline	\varname{EXPRESSION\_SIX} 		&53	&53	& 	& 	&	&53	& 	&53		&53		&53		&53	\\
\hline	\varname{EXPRESSION\_SIX\_V} 	&54	&54	&54	& 	&	&	& 	&		&		&		&	\\
\hline	\varname{EXPRESSION\_SIX\_F} 	&56	&57	&58	& 	&	&	& 	&		&		&		&	\\
\hline	\varname{EXPRESSION\_SEVEN} 	&59	&59	& 	& 	&	&59	& 	&59		&59		&59		&59	\\
\hline	\varname{EXPRESSION\_SEVEN\_V} 	&61	&61	&61	&60	&60	&	& 	&		&		&		&	\\
\hline	\varname{EXPRESSION\_SEVEN\_F} 	&	&	&	&62	&63	&	& 	&		&		&		&	\\
\hline	\varname{EXPRESSION\_EIGHT} 	&65	&65	&	&	&	&65	& 	&64		&64		&64		&64	\\
\hline	\varname{EXPRESSION\_EIGHT\_F} 	&68	&67	&	&	&	&66	& 	&		&		&		&	\\
\hline	\varname{EXPRESSION\_NINE} 		&	&	&	&	&	&	& 	&70		&70		&70		&70	\\
\hline
\end{longtable}                 
\end{landscape}

\section{Modifications}

	Nous sommes partis de la grammaire originale de l'énoncé et l'avons
	modifiée au fur et à mesure pour la rendre $LL_1$ (après les transformations
	automatiques habituelles).

	\subsection{Appels de fonctions}
		Toutes les fonctions (y compris les fonctions prédéfinies) doivent être
		appelés en précédant leur nom par un ``$\&$''. De plus, nous avons enlevé la
		possibilité d'omettre les parenthèses autour des listes	d'arguments lors de
		l'appel d'une fonction. En effet ceci ne permettait pas d'obtenir une
		grammaire $LL_1$, car un appel de fonction est une expression du plus bas
		niveau, mais le dernier argument est une expression du niveau le plus haut, et
		comme le dernier argument d'une fonction est potentiellement la derniere
		variable/le dernier token dans cette fonction (quand il n'y a pas de
		parenthèses pour entourer la liste d'arguments), l'expression de plus haut
		niveau ``hérite'' du follow de celle de plus bas niveau, ce qui crée des
		conflits.
	
	\subsection{Not}
		Le symbole ``not'' en toutes lettres tel que décrit dans la syntaxe génère le
		même genre de conflits. Ce symbole ``transforme'' ce qui se trouve derriere en
		une expression de plus haut niveau (et ce afin de respecter sa priorité
		faible). Mais ceci place une expression de haut niveau a la fin d'une
		expression de bas niveau, et on obtient des conflits. Nous avons tout
		simplement supprimé ce symbole.
		
	\subsection{Assignation} 
		La grammaire autorise l'assignation de n'importe quelle
		expression à n'importe quelle autre, c'est lors de l'analyse sémantique
		que la validité de ce genre d'expressions est déterminée.
		
	\subsection{Divers}
		\begin{itemize}
		\item On ne respecte pas les specificités de Perl comme par exemple ``0 but
		true'' qui est évalué à zéro comme entier mais à vrai comme booléen.
		\item La fonction scalar ne fait rien dans le cadre de ce projet, car nous
		n'avons pas de tableaux.
		\end{itemize}

\section{Programme et Mode d'emploi}

	Nous n'avons pas pu finir le compilateur. Les analyses lexicale et syntaxique
	fonctionnent correctement et selon les modifications décrites dans la section
	précédente, mais la génération du code n'est pas finie. Le compilateur est
	capable de générer du code ARMV5TE correct pour les définitions et appels de
	fonctions avec un nombre quelconque de paramètres, les litéraux (réels, entiers
	et chaines de caractères), certains opérateurs, les variables globales. Les
	fonctions print et length on été implémentées (nous avons le code assembleur
	des autres mais l'intégrer dans le compilateur n'est pas rapide), mais pas les
	conditions (si un code perl avec des if/elsif/else est lu en input, le comportement du programme est imprévisible).
	
	Le compilateur a été écrit en C++. Pour le lancer, il suffit de lui donner en
	argument le nom d'un fichier contenant du code perl simplifié. L'output se
	fera dans le fichier ``code.s''.
	Nous avons rédigé un makefile très simple pour le compiler.
	
\end{document}